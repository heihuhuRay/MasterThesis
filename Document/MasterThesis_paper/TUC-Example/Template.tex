%--------------------------------------------------------------------------------------%--------------------------------------------------------------------------------------
%
%  Global settings, dont change it! (excapt additional \usepackage commands)
%  Always use PDFLatex!
%
%--------------------------------------------------------------------------------------%--------------------------------------------------------------------------------------
\documentclass[a4paper, 12pt, oneside, BCOR1cm,toc=chapterentrywithdots]{scrbook}

\usepackage{graphicx}           % use for pdfLatex
\usepackage{makeidx} % f\"{u}r Benutzung des Befehls \printindex
\usepackage[colorlinks=false]{hyperref}
\usepackage{tocbibind}
\usepackage{blindtext}
\usepackage{subfigure} 
\usepackage{acronym}

\hypersetup{%
bookmarksnumbered=true, hyperindex=true,
%
%Im Acrobat Reader Subtitel 1. Ebene anzeigen
bookmarksopen=true, bookmarksopenlevel=1,
%
pdfborder=0 0 0 % Keine Box um die Links!
}

% --------------------------------------------------------------
% Force Tables and List to be added in Table of Content
% --------------------------------------------------------------

\renewcommand*{\tableofcontents}{%
  	\begingroup
  	\tocsection
  	\tocfile{\contentsname}{toc}
  	\endgroup
}
\renewcommand*{\listoffigures}{%
  	\begingroup
  	\tocsection
  	\tocfile{\listfigurename}{lof}
  	\endgroup
}
\renewcommand{\listoftables}{
	\begingroup
	\tocsection
	\tocfile{\listtablename}{lot}
	\endgroup
}
\begin{document}

%--------------------------------------------------------------------------------------%--------------------------------------------------------------------------------------
%
%  Here starts the userspace !
%
%--------------------------------------------------------------------------------------%--------------------------------------------------------------------------------------

%--------titlepage
\begin{titlepage}

{
    \begin{center}
        \raisebox{-1ex}{\includegraphics[scale=1.5]{TU_Chemnitz_positiv_gruen.pdf}}\\
    \end{center}
    \vspace{0.5cm}
}

\begin{center}

\LARGE{\textbf{Title of the master thesis}}\\
\vspace{1cm}


\Large{\textbf{Master Thesis}}\\ 
\vspace{1cm}
Submitted in Fulfilment of the\\
Requirements for the Academic Degree\\
M.Sc.\\
\vspace{0.5cm}
Dept. of Computer Science\\
Chair of Computer Engineering
\end{center}
\vspace{3cm}
Submitted by: Max Mustermann\\
Student ID: 111222\\
Date: 12.12.1212\\
\vspace{0.3cm}\\
Supervising tutor: Prof. Dr. W. Hardt \\
(further supervisors)

\end{titlepage}

%---------------------------------------------------------
% Abstract
%---------------------------------------------------------

\addchap*{Abstract}
\blindtext
\\\\
\textbf{Keywords: Keyword1, Keyword2, Keyword3, ...max 5}

%---------------------------------------------------------
% Table of Contents, List of figures, List of Tables
%---------------------------------------------------------

\tableofcontents
\listoffigures
\listoftables

%---------------------------------------------------------
% List of Abbreviations
%---------------------------------------------------------
\twocolumn
\addchap{List of Abbreviations}
\begin{acronym}[Bash]
 \acro{KDE}{K Desktop Environment}
 \acro{SQL}{Structured Query Language}
 \acro{Bash}{Bourne-again shell}
 \acro{JDK}{Java Development Kit}
 \acro{VM}{Virtuelle Maschine}
 \acro{I2C}[I²C]{Inter-Integrated Circuit}
 \acro{KDE}{K Desktop Environment}
 \acro{SQL}{Structured Query Language}
 \acro{Bash}{Bourne-again shell}
 \acro{JDK}{Java Development Kit}
 \acro{VM}{Virtuelle Maschine}
 \acro{I2C}[I²C]{Inter-Integrated Circuit}
 \acro{KDE}{K Desktop Environment}
 \acro{SQL}{Structured Query Language}
 \acro{Bash}{Bourne-again shell}
 \acro{JDK}{Java Development Kit}
 \acro{VM}{Virtuelle Maschine}
 \acro{I2C}[I²C]{Inter-Integrated Circuit}
 \acro{KDE}{K Desktop Environment}
 \acro{SQL}{Structured Query Language}
 \acro{Bash}{Bourne-again shell}
 \acro{JDK}{Java Development Kit}
 \acro{VM}{Virtuelle Maschine}
 \acro{I2C}[I²C]{Inter-Integrated Circuit}
 \acro{KDE}{K Desktop Environment}
 \acro{SQL}{Structured Query Language}
 \acro{Bash}{Bourne-again shell}
 \acro{JDK}{Java Development Kit}
 \acro{VM}{Virtuelle Maschine}
 \acro{I2C}[I²C]{Inter-Integrated Circuit}
 \acro{KDE}{K Desktop Environment}
 \acro{SQL}{Structured Query Language}
 \acro{Bash}{Bourne-again shell}
 \acro{JDK}{Java Development Kit}
 \acro{VM}{Virtuelle Maschine}
 \acro{I2C}[I²C]{Inter-Integrated Circuit}
\end{acronym}

\onecolumn
%---------------------------------------------------------
% Here starts the real work
%---------------------------------------------------------

\chapter{Introduction}
\section{Robot}

\section{NAO}

\section{CPG}  % Load Data from File intro.tex

\chapter{Tables}
% Example of, how to use a Table

\blindtext[1]

\begin{center}
\begin{tabular}{|c|c|c|c|}
\hline
Wert 1 & Wert 2 & Wert 3 & Wert 4\\
in Einheit1 & in Einheit2 & in Einheit3 & in Einheit4 \\
\hline
192 & 80& 0.3153 & 0.4900\\
500 & 120& 0.1229& 0.1787\\
1000 & 120& 0.0680& 0.0880\\
2000 & 120& 0.0361& 0.0441\\
5000 & 140& 0.0256& 0.0305\\
5000 & 164& 0.0343& 0.0880\\
\hline
\end{tabular}
\captionof{table}{This is the caption of the table}
\label{tab:table1}
\end{center}

\blindtext[3] % Load Data from File example_tables

\chapter{Figures}
% Example of, how to use figures

\blindtext

\begin{figure}[h]
\centering
\includegraphics[width=0.6\textwidth]{TU_Chemnitz_positiv_gruen.pdf}
\caption{Graphic 1}
\label{fig:pic0}
\end{figure}

\blindtext
\blindtext

\begin{figure}[h]
    \subfigure[This is the first graphic]{\includegraphics[width=0.49\textwidth]{TU_Chemnitz_positiv_gruen.pdf} \label{fig:pic1}}
    \subfigure[This is the secound graphic]{\includegraphics[width=0.49\textwidth]{TU_Chemnitz_positiv_gruen.pdf}\label{fig:pic2}}
\caption{This is the caption of the whole graphic}
\end{figure}

\blindtext

 % Load Data from File example_figures

\chapter{Referencing}
% Alternativ just write your text under \chapter like this example

\blindtext \cite{autorenrichtlinien}

\blindtext \footnote{Here is an area for your Notes}

\blindtext \footnote{\cite{lnilatex} Seite 11}
\blindtext \cite{lnilatex}
\blindtext \cite{autorenrichtlinien,pepper1992grundlagen,chen2001audiovisual}


\chapter{Subchapter}

\section{sub 1}
\blindtext[3]
\section{sub 2}
\blindtext[3]
\subsection{sub 2.1}
\blindtext[3]

\subsection{sub 2.2}
\blindtext[3]

%---------------------------------------------------------
% bibliography based on Springer Design
%---------------------------------------------------------

\bibliographystyle{splncs03}
\bibliography{bibliography}

\printindex

\end{document}
